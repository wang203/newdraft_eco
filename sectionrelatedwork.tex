\section{Related work}

Mining crowd-sourcing data from public social media data has recently
been investigated for a wide variety of applications, from predicting
election outcomes~\cite{you2015multifacetedelections}, to quantifying
tourism in natural areas from geo-tags~\cite{wood2013usingtourism}, to
predicting the stock market from online sentiment
analysis~\cite{bollen11twitter}, to inferring land
use~\cite{vsecerov2015analysis}. The vast majority of this work is
based on textual tag analysis, even for work that has studied photo
collections~\cite{you2015multifacetedelections,vsecerov2015analysis,wood2013usingtourism}.
For example, Zhang et al~\cite{ecology2012www} use geo-tagged,
time-stamped Flickr photos to estimate snow and vegetation coverage,
as we do here, but base their analysis on searching for snow-related
textual tags, which they point out is limited by how well
photographers tag their photos, and leads to false alarms
(e.g.\ photos of ``snowy owls'').  We explore the much more difficult
but potentially more accurate approach of using image analysis to
extract the semantic content automatically.

Some recent work has begun to use computer vision to extract
information about the environment. Webcam and other video from
long-term fixed cameras is particularly attractive because it provides
dense data across time, which allows the computer vision algorithms to
cue on visual changes.  For example, Laffont et
al~\cite{laffont2014transient} investigate detecting many transient
attributes of scenes over time, Glasner et al~\cite{glasner2015hot}
predict temperature, Murdock et al~\cite{murdock2015building, murdock}
estimate cloud cover, and Fedorov et al~\cite{feorov2015snowwatch,
  fedorov2014snow} estimate mountain peak snow coverage.  Other work
has used data like Google Street View, which gives very dense,
systematic spatial coverage of a place, but very sparse temporal
coverage. For example, Khosla et al~\cite{khosla2014looking} estimate
properties like neighborhood crime rates, while
Porzi~\cite{porzi2015predicting} estimate perceptions of places (like
perceived crime rates).

Automated visual analysis of time-stamped, geo-tagged social photos on
sites like Flickr and Panoramio represent another promising,
complementary data source: they are not limited to street images on
particular days as with StreetView or where a stationary webcam has
been installed, but instead can provide observations whenever and
wherever photographers choose to go. However, a major challenge is that these
user-contributed collections are much much noisier, with unconstraint
content and frequently erroneous time-stamps and geo-tags.  Leung and
Newsam~\cite{Leung:2010wa} apply scene recognition to try to classify
land use of places, while Zhou et al~\cite{zhou2014} and Lee et
al~\cite{geoinformatics2015wacv} estimate demographic and other
properties of places. These papers have studied fixed properties
of places instead of the changing environmental properties we consider here,
and are typically evaluated on local or regional scales in contrast to
our continental-scale evaluation.

The closest paper to our work is Wang et al~\cite{wang2013observing},
who like us use Flickr data to try to recognize snowfall in images.
Their results were quite preliminary, however, and used relatively
simple visual features like color histograms and GIST. Here we apply
cutting edge deep-learning based classifiers to this problem, and evaluate
on a large scale with millions of images on thousands of times and places.



%% Our work take the advantage of studying ecology phenomena with 
%% %%%%%%%%%%%%%%%%%%%%%%%%%%%%%%%%%%%%%%%%%%%%%%%%%%%%%%%%%%%%%%%%%%%%
%% %\fxnote{easy to get, more reliable}
%%  satellite maps as
%%  ground truth and use social media data to \fxnote{monitor? insight?} these information from 
%% %%%%%%%%%%%%%%%%%%%%%%%%%%%%%%%%%%%%%%%%%%%%%%%%%%%%%%%%%%%%%%%%%%%% 
%% % \fxnote{locations more related to people}
%%  . We provide continental scale quantitative evaluation and introduce our
%%  method to tackle the problem of noisy and biased data, in order to support extended studies in other areas. 
%% %%%%%%%%%%%%%%%%%%%%%%%%%%%%%%%%%%%%%%%%%%%%%%%%%%%%%%%%%%%%%%%%%%%%
%% % \fxnote{just want to say more areas in natural or not only natural but also social}

