\section{Related work}
In last few years, crowd-sourcing data from social media 
as a large scale and free to public data source
has \fxnote{received lots of attention from; or 
(become more and more popular to)} 
researchers working on using textual contents to
predict elections~\cite{you2015multifacetedelections},
using geo-tags to quantify
tourism in nature area using geo-tag profile of social media users ~\cite{wood2013usingtourism},
\fxnote{talk a lot about motivation of scientific report paper since it's in nature area}
to draw coastline~\cite{omori2014can},
%~\cite{Can geo-tags on flickr draw coastlines?}, 
using geo and temporal tags to analyze people's event-based activity 
when large group of people gathering together during a function of time 
such as football match,
and using both geo and textual tags to 
extract land use information from Panoramio~\cite{vsecerov2015analysis, oba2014towards}, 
%~\cite{Analysis of panoramio photo tags in order to extract land use information, Towards Better Land Cover Classification Using Geo-tagged Photographs},
in~\cite{ecology2012www}, 
%\fxnote{cite{www}}
Zhang \etal estimates snowfall and vegetation coverage 
based on geo, temporal and textual tags of Flickr images. 
%\fxnote{accuracy of geo-temporal problem, and now it's getting better.}

Since public-sharing photos provides such a huge potential in social and environmental study, it's 
natural to see a lot of works start analyzing image contents. Webcam providing dense temporal images 
is a good source to monitor the nature. A series of works explore sequences 
of webcam images describing outdoor scene with 40 transient attributes~\cite{laffont2014transient}, estimating dynamic cloud maps ~\cite{murdock2015building, murdock}, exploring interactions between visual elements and the temperature \fxnote{or just as the title: exploring correlations between appearance and temperature} ~\cite{glasner2015hot}, 
and monitoring the dynamic snow phenomena at mountain areas ~\cite{fedorov2015snowwatch, fedorov2014snow}. 
To evaluate the study of temperature, cloud, and snowfall amount,
 researchers can easily compare their results with satellite maps. 
Some works also use crowd-sourcing data from other sources, for example, Google street view provides selectively dense geo distributed images to help navigating the environment ~\cite{khosla2014looking} and understanding urban scene and predicting urban perception ~\cite{porzi2015predicting}, and Li \etal use the co-occurrence statistics of celebrities appears on news images to auto tag photographs of celebrity community ~\cite{li2015celebritynet}
%\fxnote{give a term like social identity?}. 
Unfortunately, 
the evaluation in these works are either not in continental scale or just via quality visualization.
Performance of social activity studies, on the other hand, are even harder to evaluate.
\fxnote{say more about our evaluation? or move this to another place?}

Flickr and Panoramio as very popular photo-sharing websites 
``involuntarily'' support researchers identifying salient city attributes and analyzing the visual similarity among different cities in order to apply computer vision to urban planning ~\cite{zhou2014recognizing}
Photo-sharing websites collecting visual contents directly from people's activity and their surrounding areas which is so important, hard to collect otherwise but also very noisy. \fxnote{write something about so there are very few work appears and so we are working on this?}

%\fxnote{not sure about keeping this paragraph or not. it seems duplicate of the 2 paragraphs above}

The fact that webcam can only be placed far away from people 
 makes it almost impossible to monitor people's activity, even not the surrounding area close to
residential or \fxnote{crowd? I mean groups of people like downtown, not ski activity but like people 
going to work and back everyday also a good topic to use temporal dense images but Webcam is not good 
at this.} Social media, on the other hand, provides a larger freedom on location distribution. 
In fact, as a complementary, almost all the photos shared online are from locations people usually go to. 
\fxnote{how helpful is this to study more areas close to urban planning, market sharing, everyday living,
anything related to people}

Our work take the advantage of studying ecology phenomena with \fxnote{easy to get, more reliable} satellite maps as
 ground truth and use social media data to \fxnote{monitor? insight?} these information from \fxnote{
locations more related to people}. We provide continental scale quantitative evaluation and introduce our
 method to tackle the problem of noisy and biased data, in order to support extended studies in other areas. \fxnote{just want to say more areas in natural or not only natural but also social}

