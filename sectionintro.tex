\section{Introduction}

Monitoring the meteorology and vegetation phenomenon is the cornerstone and key challenge of ecology and biology research. Expensive satellite images give large scale data but are limited by cloud cover, 
atmospheric conditions, struggle with fine-grained localization such as flower species distribution and are not applicable on observing human interaction with nature; 
% and direct targeting of them, 
% \cite{China tests anti-satellite weapon, unnerving U.S., Pentagon is confident missile hit satellite tank}
% need new or more deficiency of satellites
while citizen science provides high quality data but is also costly and is very difficult to practice over large scale areas.
The enormous popularity of photo-sharing website collects images in large spatial scale, from under clouds and in close focus (compare to aerial surveillance), 
moreover, they are freely accessible to the public.
The more than 300 million images uploaded to social media every day ~\cite{facebookstatistics}
 potentially contain not only human activities, but also outdoor ecology and biology 
information intentionally and incidentally as shown in Figure ~\ref{fig:flickrexp}.
% !! extend

\begin{figure}[t]
{\tiny{
\begin{center}
\begin{tabular}{@{}c@{\,\,\,}c@{\,\,\,}c@{\,\,\,}c@{\,\,\,}c@{\,\,\,}c@{\,\,\,}c@{\,\,\,}c@{\,\,\,}}
\includegraphics[width=0.05\textwidth, height=0.3in]{image/citysnow.jpg} &
\includegraphics[width=0.05\textwidth, height=0.3in]{image/citysnow2.jpg} &
\includegraphics[width=0.05\textwidth, height=0.3in]{image/dogsnow.jpg} &
\includegraphics[width=0.05\textwidth, height=0.3in]{image/humansnow.jpg} &
\includegraphics[width=0.05\textwidth, height=0.3in]{image/intentiongreen.jpg} &
\includegraphics[width=0.05\textwidth, height=0.3in]{image/waterfallgreen.jpg} &
\includegraphics[width=0.05\textwidth, height=0.3in]{image/dogtree.jpg} &
\includegraphics[width=0.05\textwidth, height=0.3in]{image/humantree.jpg} \\
\end{tabular}
\end{center}
}}
\caption{Flickr image examples capture snow and greenery evidence on purpose and as background.}
\label{fig:flickrexp}
\end{figure}


The idea of reproducing satellite maps has become more and more interesting to scientists applying 
textual mining on~\cite{bollen11twitter,ecology2012www,you2015multifacetedelections,wood2013usingtourism} 
%\fxnote{fixed,  cite{stock, ecology, election, tourists} }, 
and recently to computer vision researches directly deriving various information directly from visual contents such as
 temperature~\cite{glasner2015hot}, dynamic status of cloud~\cite{murdock}, snow coverage on mountain peak~\cite{fedorov2015snowwatch, fedorov2014snow}.
%\fxnote{cloud is fixed, cite{temperature, cloud, mountain peak}} 
In this paper, we test the feasibility of leveraging these noisy and biased images as a new 
source to observe nature. We study 2 particular phenomena, snowfall and vegetation 
coverage as they are fundamental topics in ecology and biology study, have relatively 
distinct appearance to recognize, have a good chance to appear in social media, 
and also have satellite maps available to serve as ground truth. Our approach is 
illustrated in Figure \ref{fig:overview}. 
% new source of monitoring ecology information
First, we collect a large hand-labeled data set of the existence or absence of ecology phenomena. 
Then, we train a classifier for each phenomenon by combining its most discriminative visual 
features and by using deep learning features. 
Finally, we collect 12 million images from entire North America over 4 years, make prediction 
on geo and temporal scale by aggregating visual evidence.

Inspried by an earlier work~\cite{ecology2012www} 
%\fxnote{ Fixed: cite{www}} 
analyzing ecology phenomenon from image tags only. We apply a new approach by 
understanding visual content of images, and run experiments on the exact same 
data set to study how vision techniques could help in social media data mining 
compared to using textual data alone. Also, to our best knowledge, among all the 
research works performing social sensing with image data, this is the first one 
providing continental scale quantitative performance evaluation.


\begin{figure}[h!]
\centering
\includegraphics[scale=0.7]{figure/flowchartWevaluation.png}
\caption{Overview of our approach. We train image classifiers(in blue) on large scale images. And by applying it to training images in 2007-2008, 
we train a likelihood model(in green) and finally make prediction by aggregating these visual evidence.}
% \fxnote{first classifier, then prediction}
\label{fig:overview}
\end{figure}

