\section{Related work}
In last few years, crowd-sourcing data from social media 
as a large scale and free to public data source
has \fxnote{received lots of attention from; or (become more and more popular to)} 
researchers working on 
\fxnote{cite{old draft text mining, google doc, social event paper, 1 word for each paper, social 
before natural}}. \fxnote{talk a lot about motivation of scientific report paper}.
In \fxnote{cite{www}}, Zhang \etal estimates geo-temporal snowfall and vegetation coverage 
based on Flickr image tags.
\fxnote{accuracy of geo-temporal problem, and now it's getting better.}

Since metadata of photos provides such a huge potential in social and environmental study, it's 
natural to see a lot of works start analyzing image contents. Webcam providing dense temporal images 
is a good source to monitor the nature. A series of works \fxnote{cite{webcam papers}} 
describes outdoor scene \fxnote{cite{transattri}}, estimates \fxnote{cite{temp}} through sequences 
of webcam images. 
To evaluate the study of temperature and cloud, snowfall amount \fxnote{cite{snow on mountain peak}},
 researchers can easily compare their results with satellite data. Unfortunatelly, 
the evaluation in these works are either not on continental scale or just via quality visualization.
Social activity study on the other hand, is way more complex to evaluate the performance.

Flickr and Panaramio \fxnote{check spell} as very popular photo-sharing websites , 
``unintentionally'' \fxnote{check spell} supports researchers studies urban perception and 
\fxnote{ cite{celebrity auto tag paper} and give a term like social identity?...} and \fxnote{cite{
city attributes paper} and explain this is more important to people}. \fxnote{cite{look beyond} using 
Google Street view}.

The fact \fxnote{foundamental loop hole} that webcam can only be placed far away from people 
 makes it almost impossible to monitor people`s activity, even not the surrounding area close to
residencial or \fxnote{crowd? I mean groups of people like downtown, not ski activity but like people 
going to work and back everyday also a good topic to use temporal dense images but webcam isn't good 
at this.} Social media, on the other hand, provides a larger freedom on location distribution. 
In fact, as a complementory, almost all the photos shared online are from locations people usually go to. 
\fxnote{how helpful is this to study more areas close to urban planning, market sharing, everyday living,
anything related to people}

Our work take the advantage of studying ecology phenomena with \fxnote{easy to get, more reliable
 ground truth} and use social media data to \fxnote{monitor? insight?} these information from \fxnote{
locations more related to people}. We provide continental scale quatetive evaluation and introduce our
 method to tackle the problem of noisy and biased data, in order to support extended studies about 
more areas. \fxnote{change this sentence..}

