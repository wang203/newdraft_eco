\documentclass[10pt]{article}

\usepackage{fixme}
\fxsetup{ status=draft, layout=inline}
\usepackage[pdftex]{graphicx}

\ExecuteOptions{letterpaper,10pt,twocolumn,oneside,final,journal}

\begin{document}
The \fxnote{a word for ``more and more popular''} social media websites 
collect a huge amount of latent visual information for environmental and ecology monitoring.
In this work, we propose to reconstruct satellite maps of ecology status 
across North America continent 
through 12 million publicly available geo-temporal tagged images.
We select snowfall and greenary vegetation coverage as important ecology phenomena with satellite 
maps available to compare with, and straight forward \fxnote{I just want to say easy.} to identify by human. 
First we examine the existance of snow and greenery respectively on all the images and 
aggregate this result according to the geo and 
temporal attributes to make continental scale prediction over each time period. 
Then we compare our results with satellite maps for each phenomenon and find that we fill the blanket 
when satellites failed to get any data at very cloudy area or when satellite data is too 
coarse to small changes \fxnote{like vegetation case, satellite data miss data at almost all of 
the transition period}. In this case, our results are very helpful for ecologists especially when 
the data during changing or transition period is all they need without a better way to find. \fxnote{is this clear?}
We also conduct experiments with fixed time or location and find our result 
as a promising complementary of satellite maps. Moreover, we also compare the performance based on visual evidence and textual mining. 
It turns out tags can describe snowfall very well but due to \fxnote{misleading of words describing 
vegetation}, it works less satisfactory on vegetation coverage estimation.




\end{document}